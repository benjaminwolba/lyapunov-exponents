%-------------------------------------------------------------------------
%	Chaos and Quantum Chaos, Benjamin Wolba
%-------------------------------------------------------------------------

\documentclass{article}

\addbibresource{Ref_Lyapunov.bib} 										    % integration of .bib-file

\title{Notes on Lyapunov Exponents}  
\author{Benjamin Wolba}


\begin{document}

\maketitle


\section{Introduction}

Lyapunov exponents aim to quantify the term "'\textbf{sensitive dependence of initial conditions}"'. Lets imagine two trajectories in phase space $\vec{x}(t)$ and $\vec{x}(t) + \vec{\delta}(t)$ that are initially separated by a tiny amount $|\vec{\delta}(0)| = 10^{-15}$, which is of the order of floating-point precision. \\
\indent Sensitive dependence on initial conditions means, that neighboring trajectories separate exponentially fast
%
\begin{equation}
|\vec{\delta}(t)| \approx |\vec{\delta}(0)| \, e^{\lambda t} 
\label{expincrease}
\end{equation}
%
given a positive \textbf{Lyapunov exponent} $\lambda$. The Lyapunov exponents sets the \textbf{time horizon} beyond which our predictions break down: Suppose the distance $a$ indicates our tolerance, i.e. the accuracy of our measurement equipment, so that if a measurement is within $a$, we consider it acceptable. The prediction becomes inaccurate for $|\vec{\delta}(t)| > a$, which happens after the time
%
\begin{equation}
t = \frac{1}{\lambda} \ln \left( \frac{a}{|\vec{\delta}(0)|} \right)
\label{timehorizon}
\end{equation}
% 
So even if we increase our accuracy and lower $|\vec{\delta}(0)|$ tremendously, our time horizon expands only by a couple of $1/\lambda$. \\
\indent This treatment so far has closely followed Strogatz book (\cite[chapter 9.3]{Strogatz1994}), which also points out a more rigorous definition of Lyapunov exponents. On the one hand the strength of exponential divergence varies somewhat along a trajectory, thus $\lambda$ should be average over various points. On the other hand for a $n$-dimensional phase space there are actually \textbf{$n$ different Lyapunov exponents}: Every point in phase space can be decomposed into $n$ different expanding or contracting directions featuring a distinct Lyapunov exponent $\lambda_k$ with $k = 1, \dots, n$. \\



\section{The Intuitive Approach}



\section{Gram-Schmidt Method}




% References
\nocite{*}
\printbibliography


\end{document}